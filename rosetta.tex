\documentclass{article}
\usepackage[utf8]{inputenc} % to be able to type unicode text directly
\usepackage{inconsolata} % for a nicer (e.g. non-courier) tt family font
\usepackage{mathptmx} % for times font

\usepackage{graphicx} % to include figures
\usepackage{hyperref,url} % to make clickable hyperlinks
\usepackage{minted} % for code insets
\usepackage{array} %
\usepackage{bbm} % for blackboard 1
\usepackage{amsfonts} % for \mathfrak and some symbols

\colorlet{darkgreen}{black!50!green}
\colorlet{ddarkgreen}{black!75!green}
\colorlet{darkred}{black!50!red}
\definecolor{ipol}{rgb}{.36,.29,.65}
%\usecolortheme[named=ipol]{structure}
\definecolor{term}{rgb}{.9,.9,.9}

\def\R{\mathbf{R}}
\def\F{\mathcal{F}}
\def\x{\mathbf{x}}
\def\u{\mathbf{u}}
\def\FFT{\mathtt{FFT}}
\def\IFFT{\mathtt{IFFT}}

\addtolength{\voffset}{-4em}
\addtolength{\textheight}{8em}
\addtolength{\hoffset}{-8em}
\addtolength{\textwidth}{12em}

\begin{document}

{\bf A Rosetta stone for vector calculus on graphs}

Let $G=(V,E)$ be a graph with $n=\#V$ vertices and $m=\#E$ edges.

There is a natural correspondence between~$G$ and the euclidean plane.

\bigskip

\setlength{\extrarowheight}{3pt}
\begin{tabular}{l|l|b{0.33\textwidth}}
\bf vector calculus & \bf graph theory & \bf octave code \\
\hline
euclidean plane $\R^2$ & graph $G=(V,E)$ &
\begin{minted}[bgcolor=term,fontsize=\footnotesize]{octave}
A = sparse(m,n); % adjacency matrix
\end{minted}
\\
point $p\in\R^2$ & vertex $p\in V$ &
\begin{minted}[bgcolor=term,fontsize=\footnotesize]{octave}
i = randi(1, n);
\end{minted}
\\
tangent vector $v_p\in T_p\R^2$ & edge $v=(p,q)\in E$ &
\begin{minted}[bgcolor=term,fontsize=\footnotesize]{octave}
j = randi(1, m);
\end{minted}
\\
scalar function $u:\R^2\to\R$ & $u:V\to\R,\qquad u\in\R^n$ &
\begin{minted}[bgcolor=term,fontsize=\footnotesize]{octave}
u = rand(1, n);
\end{minted}
\\
vector field $\mathbf{v}:\R^2\to\R^2$ & $\mathbf{v}:E\to\R,\qquad\mathbf{v}\in\R^m$&
\begin{minted}[bgcolor=term,fontsize=\footnotesize]{octave}
v = rand(1, m);
\end{minted}
\\
$\mathfrak{X}^0\quad$ \color{gray}(functions) & $\R^n$ & \\
$\mathfrak{X}^1\quad$ \color{gray}(fields) & $\R^m$ & \\
gradient operator $\nabla:\mathfrak{X}^0\to\mathfrak{X}^1$ &
incidence matrix $B:\R^n\to\R^m$ &
\begin{minted}[bgcolor=term,fontsize=\footnotesize]{octave}
B = incidence(A);
\end{minted}
\\
divergence operator $\mathrm{div}:\mathfrak{X}^1\to\mathfrak{X}^0$ &
transposed incidence $-B^T:\R^m\to\R^n$ &
\begin{minted}[bgcolor=term,fontsize=\footnotesize]{octave}
-B'
\end{minted}
\\
laplacian operator $\Delta:\mathfrak{X}^0\to\mathfrak{X}^0$ &
laplacian matrix $L:\R^n\to\R^n,\quad L=-B^TB$ &
\begin{minted}[bgcolor=term,fontsize=\footnotesize]{octave}
L = -B'*B;
\end{minted}
\\
$\ ?$ &
	centering operator $C:\R^n\to\R^m,\quad C=\frac{1}{2}|B|$ &
\begin{minted}[bgcolor=term,fontsize=\footnotesize]{octave}
C = abs(B)/2;
\end{minted}
\\
$\ ?$ &
recentering operator $C^T:\R^m\to\R^n$ &
\begin{minted}[bgcolor=term,fontsize=\footnotesize]{octave}
C'
\end{minted}
\\
unit smoothing $g_1*:\mathfrak{X}^0\to\mathfrak{X}^0$ &
smoothing operator $C^TC:\R^n\to\R^n$ &
\begin{minted}[bgcolor=term,fontsize=\footnotesize]{octave}
C'*C
\end{minted}
\\
directional derivative $\nabla f(a)\cdot(e_a)$ &
$(Bf)(e)$ &
\begin{minted}[bgcolor=term,fontsize=\footnotesize]{octave}
(B*f)(e)
\end{minted}
\\
domain $\Omega\subseteq\R^2$ & $\Omega\subseteq V,\quad\mathbbm{1}_\Omega\in\R^n$ &
\begin{minted}[bgcolor=term,fontsize=\footnotesize]{octave}
M = O>0;
\end{minted}
\\
oriented boundary $\partial\Omega\subseteq\R^2$ & $\partial\Omega=E\cap(\Omega\times\Omega^c)\subseteq E,\quad \mathbbm{1}_{\partial\Omega}=-B\mathbbm{1}_\Omega\in\R^m$ &
\begin{minted}[bgcolor=term,fontsize=\footnotesize]{octave}
dM = -B*M;
\end{minted}
\\
integral over a domain $\int_\Omega f$ &
${\mathbbm{1}_\Omega}^T f = \left\langle\mathbbm{1}_\Omega,f\right\rangle_{\R^n}$ &
\begin{minted}[bgcolor=term,fontsize=\footnotesize]{octave}
M'*f
\end{minted}
\\
flux accross a curve $\int_{S}\mathbf{v}\cdot\mathbf{ds}$ &
${\mathbbm{1}_S}^T\mathbf{v} = \left\langle\mathbbm{1}_S,\mathbf{v}\right\rangle_{\R^m}$ &
\begin{minted}[bgcolor=term,fontsize=\footnotesize]{octave}
S'*v
\end{minted}
\\
$\int_{\partial\Omega}\mathbf{v}=\int_\Omega\mathrm{div}(\mathbf{v})$ &
$\left\langle -B\mathbbm{1}_{\Omega},\mathbf{v}\right\rangle=\left\langle\mathbbm{1}_\Omega,-B^T\mathbf{v}\right\rangle$ &
\begin{minted}[bgcolor=term,fontsize=\footnotesize]{octave}
(-B'*M)'*v == M'*(-B'*v)
\end{minted}
\\
pointwise product $fg$ & Hadamard product $f\odot g$ &
\begin{minted}[bgcolor=term,fontsize=\footnotesize]{octave}
f .* g == diag(f) * g
\end{minted}
\\
pointwise product $f\mathbf{g}$ & Hadamard product $(Cf)\odot g$ &
\begin{minted}[bgcolor=term,fontsize=\footnotesize]{octave}
(C*f).*g
\end{minted}
\\
$\nabla(fg)=f\nabla g+g\nabla f$ &
$B(f\odot g)= Bf\odot Cg+Cf\odot Bg$ &
\begin{minted}[bgcolor=term,fontsize=\footnotesize]{octave}
B*(f.*g) == (B*f).*(C*g) + (C*f).*(B*g)
\end{minted}
\\
$\mathrm{div}(f\mathbf{g})=f\mathrm{div}(\mathbf{g})+\nabla f\cdot\mathbf{g}$ &
$-B^T(Cf\odot\mathbf{g})= f\odot C^T(-B^T\mathbf{g})+C^T\left(Bf\odot\mathbf{g}\right)$ &
\begin{minted}[bgcolor=term,fontsize=\footnotesize]{octave}
-B'*((C*f).*g) == ...
\end{minted}
\\
&&\\
elliptic PDE $A[u]=f$ &
linear system $Au=f$ &
\begin{minted}[bgcolor=term,fontsize=\footnotesize]{octave}
u = A\f;
\end{minted}
\\
parabolic PDE $u_t=A[u]$ &
first order linear ODE system $u_t=Au$ &
\begin{minted}[bgcolor=term,fontsize=\footnotesize]{octave}
u = expm(t*A)*u0;
\end{minted}
\\
$\int\|\nabla u\|^2$ &
$(Bu)^T(Bu)=u^TB^TBu$ &
\begin{minted}[bgcolor=term,fontsize=\footnotesize]{octave}
u'*B'*B*u
\end{minted}
\\
anisotropy $\mathrm{div}(D\mathbf{v})$ &
$-B^TDB\mathbf{v},\qquad D\in\mathcal{M}_{2,2}$ &
\begin{minted}[bgcolor=term,fontsize=\footnotesize]{octave}
$-B'*D*B*v$
\end{minted}
\\
\end{tabular}

\end{document}
% vim:sw=4 ts=4 spell spelllang=en:
